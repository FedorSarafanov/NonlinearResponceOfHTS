\input{text/preambule.tex}
\newcommand{\sq}[1]{\tikz{\draw[draw=#1,fill=#1] (0,0) rectangle (0.7em,0.7em);}}
\usepackage{xcolor}
\definecolor{ochre}{HTML}{e2431e} % #e2431e 0
\definecolor{lightorange}{HTML}{e7711b} % #e7711b 1
\definecolor{lightyellow}{HTML}{f1ca3a} % #f1ca3a 2
\definecolor{lightgreen}{HTML}{6f9654} % #6f9654 3
\definecolor{osci}{HTML}{82FF27}%#82FF27
\definecolor{sky}{HTML}{1c91c0} % #1c91c0 4
\definecolor{violet}{HTML}{43459d} % #43459d 5

\begin{document}  
% \end{document}
%%%%%%%%%%%%%%%%%%%%%%%%%%%%%%%%%%%%%%%%%%%%%%%%%%%%%%%%%%%%%
\begin{frame}[plain]
	\centering
	\vspace{1.2cm}
	\begin{beamercolorbox}[sep=8pt,center]{title}
		\bf\usebeamerfont{title}\inserttitle
	\end{beamercolorbox}
	\vspace{0.5cm}
	\normalsize \textbf{Работу выполнили:}\\
	\large
	\underline{Сарафанов Ф.Г.}, %
	{Платонова М.В.}, %
	{Новиков А.Г.}
	% Рогов М.А. #wasted
	% Геликонова В.Г. % #wasted
	\\ 
	\vspace{0.5cm}
	\normalsize{\textbf{Научный руководитель:}\\}
	\large{Пестов Е.Е.}
	% \large{Яковлев А.И.}
	% \large{Антипов О.Л.}
	% \large{Щапин Д.С.}
	% \large{Мареев Е.А.}
	\vfill
	\small{Нижний Новгород -- 2018}
\end{frame}
%%%%%%%%%%%%%%%%%%%%%%%%%%%%%%%%%%%%%%%%%%%%%%%%%%%%%%%%%%%%%
% \begin{frame}[t]s
% 	\frametitle{Содержание}
% 	% \fontsize{6pt}{7.2}\selectfont
% 	\setbeamerfont{subsection in toc}{size=\tiny}
% 	\setbeamerfont{section in toc}{size=\tiny}
% 	\tableofcontents
% \end{frame}
%%%%%%%%%%%%%%%%%%%%%%%%%%%%%%%%%%%%%%%%%%%%%%%%%%%%%%%%%%%%%

\section{Введение}
\subsection{Цели работы}
\begin{frame}[t]
	\frametitle{Цели работы}
	% \textbf{Цели}\\
	\vfill
	\begin{spacing}{1}
		\begin{enumerate}
			\item Ознакомиться с моделью нейрона, обладающей свойствами генерировать  подпороговые колебания и импульсы возбуждения
			\item Феноменологически получить модельные уравнения и качественно исследовать их динамику
			\item Рассмотреть электронную схему, соответствующую модельным уравнениям
			\item Осуществить компьютерный и физический эксперименты, сравнить результаты
			% феноменологическая модель
		\end{enumerate}
	\end{spacing}
	\vfill
\end{frame}

\subsection{Сверхпроводимость}
\begin{frame}[t]%[bg]
	\frametitle{Основные свойства сверхпроводимости}
	\vspace{-1.5em}
	% \vspace{-0.5em}
	\begin{columns}
		\begin{column}{0.49\textwidth}%\centering
			\begin{figure}[h]
				% \hspace{-2em}
				\includegraphics[width=0.5\linewidth]{img/messner}
				% \vspace{-0.5em}
				\caption{Эффект Мейснера}
			\end{figure}
			\vspace{-2em}
			\begin{figure}[h]
				% \hspace{-2em}
				% \includegraphics[width=0.8\linewidth]{pic/rt}
				\includegraphics[width=0.8\linewidth]{pic/rt2}
				\vspace{-0.5em}
				\caption{Падение сопротивления до 0}
			\end{figure}
		\end{column}
		\begin{column}{0.49\textwidth}%\centering
			\begin{figure}[h]
				% \hspace{-2em}
				\vspace{-1em}
				\includegraphics[width=0.8\linewidth]{pic/ns1}
				\vspace{-0.5em}
				\caption{Сверхпроводники 1-го рода}
			\end{figure}
			\vspace{-2em}
			\begin{figure}[h]
				% \hspace{-2em}
				% \includegraphics[width=0.8\linewidth]{pic/rt}
				\includegraphics[width=0.8\linewidth]{pic/ns2}
				\vspace{-0.5em}
				\caption{Сверхпроводники 2-го рода}
			\end{figure}
		\end{column}
	\end{columns}		
\end{frame}

\subsection{Механизмы нелинейности сверхпроводников}
\begin{frame}[c]%[bg]
	\frametitle{Механизмы нелинейности сверхпроводников}
	% \vspace{-1em}
	% \vspace{-0.5em}
	\begin{enumerate}
		\item Нелинейность Гинзбурга-Ландау
		\item Тепловая нелинейность $n_s=n_s(T)$
		% Зависимость концентрации сверхпроводящих 
		% электронов от температуры
		\item Вихревая нелинейность
		\item Нелинейность Джосефсона
	\end{enumerate}
\begin{center}
	$\Downarrow$
\end{center}
% В приближении слабого сигнала выполняется феноменологическая формула нелинейности:
\begin{equation*}
	\vec{j}_s\sim\vec{A}\left(1-\frac{A^2}{A_c^2}\right)
\end{equation*}

	% \begin{columns}
	% 	\begin{column}{0.49\textwidth}%\centering
	% 		\begin{figure}[h]
	% 			% \hspace{-2em}
	% 			\includegraphics[width=0.86\linewidth]{img/above.png}
	% 			\caption{СВЧ-зонд над образцом с регулируемой температурой}
	% 			% \vspace{1em}
	% 		\end{figure}
	% 	\end{column}
	% 	\begin{column}{0.49\textwidth}%\centering
	% 		\begin{figure}[h]
	% 			% \hspace{-2em}
	% 			\includegraphics[width=0.86\linewidth]{pic/exp}
	% 			\caption{Конструкция ближнепольного СВЧ-зонда}
	% 		\end{figure}
	% 	\end{column}
	% \end{columns}		
\end{frame}

\subsection{Диаграмма температурных зависимостей}
\begin{frame}[c]%[bg]
	\frametitle{Диаграмма температурных зависимостей}
	% \vspace{-1em}
	% \vspace{-0.5em}
	\begin{figure}[H]
		\centering
		\includegraphics[scale=1.2]{pic/img_3a}
		% \caption{Caption here}
		\label{fig:chem}
	\end{figure}	
\end{frame}

\subsection{Блок-схема экспериментальной установки}
\begin{frame}[c]%[bg]
	\frametitle{Блок-схема экспериментальной установки}
	% \vspace{-1em}
	\vspace{-0.5em}
	\begin{figure}[H]
		\centering
		\includegraphics[]{pic/chem}
		% \caption{Caption here}
		\label{fig:chem}
	\end{figure}	
\end{frame}

\subsection{СВЧ-зонд}
\begin{frame}[c]%[bg]
	\frametitle{СВЧ-зонд}
	% \vspace{-1em}
	% \vspace{-0.5em}
	\begin{columns}
		\begin{column}{0.49\textwidth}%\centering
			\begin{figure}[h]
				% \hspace{-2em}
				\includegraphics[width=0.86\linewidth]{img/above.png}
				\caption{СВЧ-зонд над образцом с регулируемой температурой}
				% \vspace{1em}
			\end{figure}
		\end{column}
		\begin{column}{0.49\textwidth}%\centering
			\begin{figure}[h]
				% \hspace{-2em}
				\includegraphics[width=0.86\linewidth]{pic/exp}
				\caption{Конструкция ближнепольного СВЧ-зонда}
			\end{figure}
		\end{column}
	\end{columns}		
\end{frame}


\subsection{Нелинейный отклик ВТСП-пленки}
\begin{frame}%[bg]
	\frametitle{Нелинейный отклик сверхпроводящей пленки}
	\begin{figure}[h]
		% \hspace{2em}
		\includegraphics[width=0.9\textwidth]{pic/figg}
	\end{figure} 
\begin{tikzpicture}[remember picture,overlay]
\node at (9,6) {\includegraphics[scale=1]{pic/plenka}};
\end{tikzpicture}
\end{frame}

% %%%%%%%%%%%%%%%%%%%%%%%%%%%%%%%%%%%%%%%%%%%%%%%%%%%%%%%%%%%%%
\section{Заключение}
\subsection{Выводы}
\begin{frame}
	\frametitle{Выводы}
	\begin{enumerate}
		\item Изучены основные свойства сверхпроводников
		\item Методом ближнепольной СВЧ-микроскопии снята нелинейная зависимость мощности отраженного от ВТСП-пленки сигнала $P_{3\omega}(T)$, с помощью которой:
		\begin{enumerate}
		\item Определена средняя критическая температура и её пространственный разброс: $$\langle T_c\rangle_{S}=88.5\text{ K},\qquad T_c\in 87..89 \text{ K}$$
		\item Сделан вывод о наличии разных фаз сверхпроводимости в исследуемой ВТСП-пленке
		\item Качественно показана неоднородность $\vec{j}_\text{кр}$ в ВТСП-пленке
		\end{enumerate}
	\end{enumerate}
\end{frame}
% %%%%%%%%%%%%%%%%%%%%%%%%%%%%%%%%%%%%%%%%%%%%%%%%%%%%%%%%%%
\subsection{Спасибо за внимание}
\begin{frame}[plain]
	\vspace{4cm}
	\begin{center}
		\Huge
		Спасибо за внимание!
	\end{center}
	\vspace{2.5cm}
	\begin{center}
		\color{black!60!white}
		Презентация подготовлена в издательской \\
		системе LaTeX с использованием пакетов \\
		PGF/TikZ и Beamer
	\end{center}
\end{frame}
\end{document}